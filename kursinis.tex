\documentclass{VUMIFPSkursinis}
\usepackage{algorithmicx}
\usepackage{algorithm}
\usepackage{algpseudocode}
\usepackage{amsfonts}
\usepackage{amsmath}
\usepackage{bm}
\usepackage{caption}
\usepackage{color}
\usepackage{float}
\usepackage{graphicx}
\usepackage{listings}
\usepackage{subfig}
\usepackage{array}
\usepackage{wrapfig}
\usepackage{tabu}

% Titulinio aprašas
\university{Vilniaus universitetas}
\faculty{Matematikos ir informatikos fakultetas}
\department{Programų sistemų katedra}
\papertype{Kursinis darbas}
\title{Detalus vaizdų panašumas naudojant trejetų tinklus}
\titleineng{Fine-grained Images Similarity using Triplet-network}
\status{4 kurso 6 grupės studentas}
\author{Andrius Butkevičius}
% \secondauthor{Vardonis Pavardonis} % Pridėti antrą autorių
\supervisor{dr. Vytautas Valaitis}
\date{Vilnius – \the\year }

% Nustatymai
% \setmainfont{Palemonas} % Pakeisti teksto šriftą į Palemonas (turi būti įdiegtas sistemoje)
\bibliography{bibliografija}
\let\[\relax \let\]\relax % avoid warnings in the log file
\DeclareRobustCommand{\[}{\begin{equation}}
\DeclareRobustCommand{\]}{\end{equation}}
\begin{document}

\maketitle
\thispagestyle{empty} 

\tableofcontents


\sectionnonum{Įvadas}

\thispagestyle{empty} 
Vaizdų panašumo įvertinimas ir lyginimas tampa vis plačiau naudojamas ir susilaukia vis didesnio dėmesio informacinių technologijų srityje.Visgi plėtojant šią technologiją ir siekiant išgauti kuo korektiškesnius rezultatus yra susiduriama su problemomis, nes kompiuteriui nėra lengva atskirti vizualius skirtumus ir objektų bruožų panašumus taip lengvai, kaip palyginus žmogui, kuris gali akimirskniu sugebėti atpažinti aplink jį esamus objektus. Vaizdų identifikavimas ar jų palyginimas plačiai taikomas šiuo metu, pvz.: vaizdinės tapatybės nustatymui, veido atpažinimui, esamos vietovės aptikimui pasitelkient beipiločio orlaivio užfiksuotas reljefo nuotraukas, vaizdų radimui pasitelkiant paieškos sistemas. Norint sugebėti tai atpažinti ir klasifikuoti, yra pasitelkiami įvairūs tinklų modeliai ir jie treniruojami. Tam panaudojama tokie modeliai kaip Siamo arba trejetų tinklai. Šie tinklai susideda iš dviejų ar trijų vienodų ir lygiagrečių konvoliucinių neurono tinkle esančių atšakų, kurios tarpusavyje dalinasi svoriais, kurių dėka galima gauti aukšto lygio nuotraukų bruožų atvaizdavimą taip leidžiant panašioms nuotraukoms būti kuo arčiau sujungtoms viena su kita funkcijų erdvėje. Tuo metu nepanašiems vaizdams – būnant kuo toliau nuo teisingų nuotraukų. 
\newline	
Rinkoje yra siūlomi įvairūs konvoliucinių neuronų tinklų modeliai, pvz.: Alexnet, VGGNet, GoogLetNet, ResNet \cite{Aerial_image_similarity}. Verta paminėti, kad visi šie modeliai turėjo efektyvius atpažinimo rezultatus ILSVRC. Taigi vis dažniau yra pasitelkiama prieš tai minėti trejetų tinklai, kurie optimizuoja bruožų atstumus erdvėje. Visgi sudėtingiausia problema išlieka dėl semantinio tarpo problemos, kuri atsiranda tarp žemos rezoliucijos nuotraukos pikselių užfiksuotų kompiuterinių sistemų ir aukšo lygio semantinių koncpetų, kurias suvokia žmonės. Dėl to reikia rasti geresnių būdų, kaip sugebėti pateikti nuotrauką kompiuteriui ir gauti gilesnias jos semantinius bruožus. 
\newline
Todėl šio darbo tikslas yra pasiūlyti pasrinktą trejetų tinklų modelį \cite{Aerial_image_similarity}, kuris sugebėtų atpažinti žmonių siluetus su pasirinktu duomenų rinkiniu. Palyginti pasirinktą modelį su kitais esamais rinkoje modeliais, palyginti rezultatus. Taip pat išskirti metrikas tam. Surasti su kuriais vaizdais pasirinktas modelis prastai vykdo atpažinimą ir kodėl.

\pagebreak
 
\sectionnonum{Uždaviniai}
\thispagestyle{empty}
\begin{enumerate}
\item{Palyginti kitų neuronų tinklų modelių architektūras.}
\item{Pasiūlyti pasirinktą trejetų tinklų modelį bei atpažinti vaizdų panašumus iš pasirinkto duomenų rinkinio.}
\item{ Išskirti metrikas pagal kurias galėtų analizuoti gautus rezultatus, panaudojant trejetų tinklų modelį.}
\item{Išskirti pasirinkto tyrimo rezultatų trūkumus ir išsiaiškinti galimas sritis ateities darbams.}
\end{enumerate}

\pagebreak
\section{Trejetų ir Siamo tinklų tyrimų literatūros analizė}

\subsection{Gilieji konvoliuciniai neuronų tinklai}
Giliajame treniravime(mokyme), šis tinklas yra klasė iš giliųjų neuronų tinklų, kuri dažniausiai naudojama pritaikant vaizdų atpažinimą. Tai yra algoritmas, kuris įvestyje pasiima vaidą (nuotrauką ar paveiksliuką), priskiria jam svorius ir jos bruožų tendencijas įvairiose įvesties dalyse ir sugeba pagal tai atskirti panašumus lyginant su kitomis įvestimis.

\subsection{Trejetų tinklai}
Trejetų tinklų modelis būvo pasiulytas 2014m., moksliniko Chang Wang \cite{Learning_fine_grained_image}. Modelio veikimo principas susideda iš trijų identiškų konvoliucinių neuroninių tinklų šakų, kurie tapusavyje dalijasi gautais svoriais. Kai trejetų tinklas įvestyje gauna tris pavyzdžius, išvestyje yra grąžinama dvejos tarpinės reikšmės, kurios nurodo atstumus lyginamus su pagrdine įvestimi ir kitomis dvejomis įvestimis. Jų rezultatas yra vektoriai. 1 pav. parodytas abstraktus treniravimosi procesas, trejetų tinklų modelio.
\begin{figure}[H]
\centering
\includegraphics[scale=0.5]{img/Triplet_network}
\caption{Trejeto tinklo treniravimosi procesas \cite{Improved_triplet_network}} % Antraštė įterpiama po paveikslėlio
\label{img:mlp}
\end{figure}
\pagebreak

\subsubsection{Architektūra}
Trejetų tinklų modelis įvedimo dalyje reikalauja trijų įvesčių tuo pačiu metu. Kiekvienas jų turi savo pavadinimą ir reikšmę tinkle, pagrindinis $x^a$, pozityvus $x^p$, neigiamas $x^n$. Įvesties poros $x^a$ ir $x^p$ yra tos pačios kategorijos arba panašūs semantiškai įvestys. Tuo metu $x^a$ ir $x^n$ yra skirtingos kategorijos arba nepanašūs vaizdai. Pasitelkiant funkcijas yra apskaičiuojama jų semantiniai panašumai atstumo erdvėje (nes jie yra pateikiami kaip vektoriai). Ši funkcija yra vadinama nuostolių funkcija, jos galima funkcija aprašyta žemiau. Kur parametras $\alpha$ rodo tarpą tarp $x^a$ ir $x^p$ bei $x^a$ ir $x^n$. $N$ reiškia skaičių nusakantį kiek trejetų įvesčių yra pateikiama. Tikslas yra pasiekti, kad $x^a$ ir $x^p$ būtų mažiau nei $x^a$ ir $x^n$ \cite{Face_recognition}.

\[loss = \sum_{x=1}^{N} max(d(x^a, x^p) - d(x^a, x^n) + \alpha, 0)\]
Kur $d$ yra atstumo metrika, taip kad $d(x^a, x^p) < d(x^a, x^n)$, o $N$ yra skaičius trejetų.

Nuostolių funkcija nusako, kaip gerai algoritmas modeliuoja pasirinktą duomenų rinkinį. Jeigu prognozė rezultatui yra netiksli, ši funkcija grąžina didesnę reikšmę. Neuroninių tinklų modeliams galima naudoti ir kitas nuostolių funkcijas, priklausomai nuo duomenų rinkinio ir norimo rezultato gauti.

\subparagraph{Alternatyvios nuostolių funkcijos}
Artima šiai funkcijai yra ši lygtis. Ji naudoja kvadratinį Euklido  \cite{Aerial_image_similarity} atstumą kaip atstumo metriką ir $\alpha > 0$, nurodanti ribą tarp dviejų vaizdų ir absoliučios sumos reikšmės.
\[loss = \sum_{x=1}^{N} [||x^a - x^p|| - ||x^a - x^n|| + \alpha]\]

\begin{figure}[H]
\centering
\includegraphics[scale=0.8]{img/Triplet_network_branchCNN}
\caption{Trejeto tinklų veikimo principas kartu su neuroninių tinklų lygiagrečiomis atšakomis} % Antraštė įterpiama po paveikslėlio
\label{img:mlp}
\end{figure}

Šis neuroninių tinklų architektūrinis sprendimas padeda apmokyti duomenų klasifikavimą išskaidant duomenis pagal panašumus ir skirtumus 2 pav.. Jų metu keli lygiagretūs gileiji neuroniniai tinklai yra apmokami bei tuo pačiu metu jie dalijasi svoriais vieni su kitu treniruojant tinklą. Trejetų tinkle tikslas yra sukurti trejetus, kurie susideda iš pagrindinio $p^a$, teigiamio $x^p$, neigiamio $p^n$ − įvesties elementų. Pagrindinis elementas, tai kažkokia įvestis (nuotrauka, paveiksliukas, muzikos įrašas ir t.t), kuriai mes bandome rasti atitikimą iš kitos įvesties. Teigiama įvestis – panašus elementas atitinkamai pagrindiniai nuotraukai. Tuo tarpu neigiamas elementas skaitosi tas, kuris neturi panašumų su pagrindine nuotrauka. Neuroninai tinklai apskaičiuoja  $\pi : f(\pi) \in$
Šie trys elementai yra įvedami nepriklausomai vienas nuo kito į  tris identiškus giliuosius neuroninius tinklus, kurie dalijasi vienoda architektūra ir parametrais. Jų metu yra apskaičiuojami atstumai tarp šių elementų. Šiam apskaičiavimui yra naudojama kaip ir prieš tai minėta nuostolių funkcija.
\newline
Įmanomi rezultatai trjetų tinkle:
\begin{itemize}
\item{Lengvas trejatas: trejetai, kurie turi nuostolį, su reikšme 0, nes $d(x^a, x^p) + \alpha < d(x^a, x^p)$}
\item{Sudėtingas trejetas: trejetai, kurių neigiamumas yra arčiau pagrindinio negu teigiamo, $d(x^a, x^p) < d(x^a, x^p)$}
\item{Pusiau sudėtingas trejetas: trejetai, kurių neigiamumas nėra arčiau teigiamo, tačiau vis tiek turi teigiamą nuostolį: $d(x^a, x^n) < d(x^a, x^p) +\alpha$}
\end{itemize}

\subsection{Siamo tinklas}
Pirma kartą publikuotas 1990 metais, autorių Bromely ir Yann LeCun, siekiant išspręsti parašo verifikavimo problema kaip vaizdo atitikimo probelmą \cite{Siamese_signature_verifiction}.
\subsubsection{Architektūra}
Tinklo modelis – neuronų tinklas turintys kelis ar daugiau identiškus neuroninių tinkle atšakas su vienodais parameterais \cite{Siamese_Network} (panašiai kaip ir trejetų tinklų modelyje). Šis modelis naudoja vienodus svorius vykdant tuo pačiu metu ir panaudojant du skirtingus įvesties vektorius, apskaičiuojant palyginamus išvesties vektorius. Šie vienodi tinklai su skiringais įvesties duomenimis yra sujungti pagal energijos funkciją $E$. Ši matematinė funkcija apdoroja kai kurias metrikas tarp labiausiai išryškintų bruožų vaizdavimų kiekvienoje pusėje.
Parametrai tarp šių vienodų tinkle yra apriboti, kad būtų vienodi. Svorių apjungimas užtikrina, kad du labai panašūs vaizdai nebūtų tarp jų atitinkamų tinkle  atšakų išvesti labai skirtinguose erdvės lokacijose. Taip pat, reikia paminėti, kad tinklas yra simetriškas, nesvarbu kuriam iš vienodų tinklų įvesime vaizdą, visada gausime tokias pačias metrikas.
Standartinė išlaidų funkcija skirta mokymosi pavyzdžiui $(x_1, x_2)$ yra pasiūlyta Hadselio. 
\[L(W,(Y, x_1, x_2)) =  \frac{1}{2}(1 − Y )(D_w)^2 + (Y) \frac{1}{2} {max(0, m − D_w)}^2\], jei $(x_1, x_2)$  yra panašios poros, ir $Y = 1$ kitu atveju. $m$ yra riba nurodanti norimą slenkstį atstumui tarp $x_1$ ir $x_2$ jeigu jie nėra panašūs. Dėl laisvo suregualiavimo, dažniausiai taip yra sunkiau ištreniruoti trejetų tinklus negu Siamo tinkle. Tarkime $x_1, x_2$ yra pora vektorių, $Y$ laikysime dvejatainę žymę, kur $Y = 1$ reiškia, kad vektoriai $x_1, x_2$ yra laikomi panašiais, $Y = 0$ - priešingu atveju. $D_w$ - parametrizuota atstumo funkcija (Euklido).

\begin{figure}[H]
\centering
\includegraphics[scale=1.0]{img/Siamese}
\caption{Siamo tiklų modelis} % Antraštė įterpiama po paveikslėlio
\label{img:mlp}
\end{figure}

\subsection{Nefiksuoto dydžio vaizdai konvoliuciniuose tinkluose}
Gilieji konvoliuciniai tinklai reikalauja fiksuoto dydžio įvesties vaizdų. Tačiau realiame gyvenime nuotraukų ar paveiksliukų dydžiai nėra fiksuoti ir varijuoja plačiai.
\newline
Jeigu yra bandoma primiktynai pakeisti į reikiama dydį, nuotrauką karpant ar deformuojant, informacija, patalpinta nuotraukose bus prarasta. To pasekoje tikslumas nuotraukų klasifikacijos ar objektų identifikavime bus sumažintas ir nepataisomai sugadintas. Nors neuroniniuose tinkluose, konvoliuciniai sluoksniai nereikaljau fiksuoto dydžio įvesties ir geba generuoti specifinius bruožus vaizdo iš bet kokio dydžio nuotraukų. Visgi, pilnai sujungti sluoksniai privalo turėti fiksuoto dydžio įvestį dėl jų pačių apibrėžimo. Dėl to apribojimas fiksuoto dydžio nuotraukų ateina tik iš pilnai sujungtų sluoksnių reikiamos ypatybės.
Viena iš šios problemos sprendimo būdų yra naudoti erdvinės piramidės talpinimą \cite{Spatial_pyramid_pooling}, tokiu būdų galima bandyti identifikuoti vaizdus, kurių rezoliucijos yra skirtingos.
\newline
Šis būdas ištraukia vaizdo bruožus iš bruožų žemėlapio (angl. map) per $4 x 4$, $2 x 2$ ir $1 x 1$ kvadratų tinklelio. Tada SPP sluoksnis pateikia $16 + 4 + 1 = 21$ skirtingus aruodus (angl. bin) ir gauna fiksuoto dydžio išvestį kviečiant kviekvieną bloką. Po erdvinės piramidės talpinimo būdo išgavimo, bet kuris bruožų žemėlapis gali generuoti 5736 dimensijų ypatybių vektorius, kur $5736 = 24 x 26$. SPP sluoksnis pasiima ypatybes ir generuoja  fiksuoto dydžio išvestis, kuris galiausiai yra perduodamas į pilnai sujungtus sluoksnius.
\pagebreak

\section{Įvertinimo funkcija}
Kelios įvertinimų metrikos yra naudojamos: panašumo tikslumas bei \emph{score-at-top-K}, kai $K = 30$. Panašumo tikslumas yra išreiškiamas procentaliai pagal tai, kiek trejetų buvo korektiškai sureitinguota. Sakykime, kad turime trejetą, su šiais įvesties parametrais $tt = (x^a, x^p, x^n)$, kur $x^p$ turėtų būti arčiau(panašesnis)šalia $x^a$. Laikant, kad $x^a$ yra įvesties užklaus,a žiūrime į rezultatus kitų įvesties duomenų. Jeigu $x^p$ yra reitinguojamas aukščiau nei $x^n$, tai tada teigiame, kad atitinkamas trejetas yra sureitinguotas teisingai. Kita mums reikalinga metrika, kuri buvo užsiminta anksčiau \emph{score-at-top-K}. Jis nusako skaičių teisingai sureitinguotų trejetų bei atimant iš jo skaičių, kuris nusako neteisingai sureitinguotus trejetus iš pogrupio trejetų, kurių reitingas yra didesnis, nei kintamasis $K$. Pogrupis yra pasirenkamas tokia tvarka: kiekvienam užklausos paveikslėliui iš duomenų aibės, ištraukia 1000 naujų paveikslėlių iš tos pačios teksto užklausos ir yra bandoma taip reitinguoti juos, pasitelkiant išmoktas metrikas. Jei trejetų reitingas yra aukštesnis negu $K$, jeigu jo $x^p$ arba $x^n$ tada yra tarp geriausiai reitinguojamų $K$ kaimynų iš užklausos su paveikslėliais $x^a$.
\pagebreak

\subsection{Pasirenkami duomenis įvertinimui analizuoti}
Kadangi darbo tema susijusi su detaliu vaizdų atpažinimuu, kuris negali būti charakterizuotas pagal vaizdų kategorijų žymes, literatūroje buvo surasta informacijos, panaudojant trejetų duomenų rinkinys norint įvertinti nuotraukų panašumus su skirtingais modeliams \cite{Learning_fine_grained_image}.
Buvo paimta 1000 populiariausių teksto užklausų su atrinktais trejetais $(x^a, x^p, x^n)$ su Google 50 paieškos rezultatų kiekvienai užklausai \cite{Learning_fine_grained_image}.

Atpažinimo efektyvumas pagal metrikas yra nurodytas 1 lentelė ir 2 lentelė. "DeepRanking", kuris  paminėtas buvo literatūroej, yra gilusis reitingavimo modelis treniruotas su 20\% neigiamais pavyzdžiais. Visgi pastebima, kad joks vaizdų atpažinimo modelis be treniravimo nepasiekia gerų rezultatų ir nėra dinamiškai prisitaikantys prie naujo duomenų rinkinio, kuris priklauso kitoms vaizdų klasifikacijos kategerijoms. L1HashKCPA pasiekė visai neblogą rezultatą su ganėtinai mažai dimensijų, tačiau jo našumas yra ganėtinai prastesnis negu "DeepRanking" metodo.

\begin{center}
\begin{tabular}{ | m{10em} | m{5em}| m {3em} |} 
\hline
Metodas & Tikslumas & Score-30 \\
\hline
ConvNet & 82.8\% & 5772 \\
\hline
Single-scare Ranking & 84.6\% & 6245 \\
\hline
OASIS on Single-scale Ranking & 82.5\%  & 6263 \\
\hline
Single-Scale $\&$ Visual Feature & 84.1\% & 6765 \\
\hline
DeepRanking & 85.7\% & 7004\\
\hline
\end{tabular}
\captionof{table}{Tikslumo metrikų rezultatai (1)}
\end{center}

\begin{center}
\begin{tabular}{ | m{10em} | m{5em}| m {3em} |} 
\hline
Metodas & Tikslumas & Score-30 \\
\hline
Wavelet & 62.2\% & 2735\\
\hline
Color & 62.3\% & 2935 \\
\hline
SIFT-like & 65.5\%  & 2863 \\
\hline
Fisher $\&$ Visual Feature & 67.2\% & 3064\\
\hline
HOG & 85.7\% & 7004\\
\hline
SPMKtexton1024max & 66.5\% & 3556\\
\hline
L1HashKPCA & 76.2\% & 6356\\
\hline
OASIS & 79.2\% & 6813\\
\hline
Golden Features & 80.3\% & 7165 \\
\hline
DeepRanking & 85.7\% & 7004\\
\hline
\end{tabular}
\captionof{table}{Tikslumo metrikų rezultatai (2)}
\end{center}

\pagebreak

\section{Trejetų tinklo vaizdų atpažinimo tyrimas ir vertinimas}
\subsection{Tikslumo metrikos}
Šiame darbe buvo naudojama tikslumo metrika (4), kuri nauodojoma dvejatainių klasifikatorių vertinime.

\[ accuracy=\frac{\sum  \ \ \textrm{Teigiamas} + \sum \ \ \textrm{Neigiamas} }{\sum \ \ \textrm{Visa populiacija} } \]

Kadangi yra labai retas atvejis, kad $d(x_a, x_p)$ būtų lygus $d(x_a, x_n)$, galima laikyti, kad $\sum \ \ \textrm{Teigiamas}$ yra lygus $\sum \ \ \textrm{Neigiamas}$. Del to, galima skaičiuoti tik $\sum \ \ \textrm{Teigiamas}$ iš visų įvesties trejetų sumos $N_t$ treniruojamame rinkinyje. Galutinė lygtis (5), kuri naudojama tikslumui lyginti.
\[ accuracy=\frac{\sum d(x_a, x_p) + \sum d(x_a, x_n) }{ N_t } \]

\subsection{Treniravimo ir testavimo aplinka}
Trejetų tinklo modelis buvo treniruoamas naudojant Lenovo Y-700 kompiuterį su šiomis specifikacijomis: CPU i7 6700K (4 branduolių), 8GB RAM, GPU Nvidia GTX 960m (4GB RAM). Norint pasileisti trejetų modelį būtent šiam kompiuteriui (priklausomai nuo mašinos tai gali skirtis) buvo atsisiųsta Nvidia CUDA programinė įranga su tikslu, kad modelio treniravimas būtų vykdomas pasitelkiant vaizdo plokštę, o ne procesorių.
\newline
Priežastis kodėl buvo pasirinkta vaizdo plokštė yra todėl, kad gilusis mokymasis yra intensyvi skaičiavimo užduotis. Į gilųjį mokymasi įeina didžiuliai matricų skaičiavimai (ypač sandauga) ir kitos operacijos, kurios gali veikti paraleliai todėl vaizdo plokštė ateina į pagalbą, nes viena vaizdo plokštė gali turėti tūkstančius branduolių, tuo metu procesorius turi žymiai mažiau branduolių, nors jie ir žymiai greitesni negu vaizdo plokštės \cite{Performance_of_GPU}.
\newline
Pasirenktas trejetų tinklų modelis yra implementuotas naudojant Tensorflow karkasą, Python 3.5.1.
\newline
Vaizdų duomenų rinkinyje buvo 3884 nuotraukos, kuriose jau buvo surikiuotos teisinga eilės tvarka, kur pirma nuotrauka pagrindinė, antra nuotrauka - teigiama (kuri yra panaši į pagrindinę) bei neigiama (nepanaši į pagrindinę nuotrauką) ir tokia eilės tvarka yra išsidėstę visos kitos nuotraukos.
Ištreniravus šiuos duomenis yra bandoma analizuoti modelio tikslumą imant vaizdų pavyzdžius iš testinio duomenų rinkinio.

\subsection{VGG16 tinklo modelis}
Šiame darbe, žmonių siluetų atpažinimui yra naudojamas trjetų tinklų modelis \cite{Aerial_image_similarity}, kurio architektūra naudoja VGG16 tinklo bazės sluoksnius klasifikavimui todėl tik labai panašūs vaizdai gali būti naudojami duomenų treniravimui tam, kad pagerinti tinklo atlikimą bei išsaugoti skaičiavimo išteklius.
\newline
VGG16 yra konvoliucinių neuroninių tinklų modelis, išrastas K. Simonian ir A. Zisserman (Oksfordo universitetas). Modelis sugeba pasiekti 92.7\% top-5 testų tikslumą (treniravimo duomenys pasiimti iš ImageNet). VGG16 buvo treniruojamas savaitę iš savaitės naudojant Nvidia Titan Black vaizdo plokščių šeimas.
\subsection{Tyrimo rezultatai}
Šiame darbe buvo bandyta atpažinti detalius vaizdus, naudojant vaizdų duomenų rinkinį CUHK01. Šiame duomenų rinkinyje yra pavaizduota 3884 viešose vietose einančių pėsčiųjų kadrai. Todėl naudojant ši duomenų rinkinį buvo analizuojama, kaip su pasirinktu trejetų modeliu seksis atpažinti skirtingus žmogaus kadrus, kurie buvo nufotografuoti įvairiai -  iš kelių pusių, iš nugaros ir priekio, esant skiringai žmogaus eisenos pozicijai, skirtingam apšvietimui ir kontrastamas, įsiterpiant kitiems objektams prie pėsčiojo.
\newline
Ištreniravus trejetų tinklą su pasirinktu duomenų rinkiniu ir pradėjus testuoti jo rezultatus, gavau 85,7\% tikslumo rezultatą. Tai yra gan aukštas įvertinimas, nes atpažinti to paties pėsčiojo skirtingus kadras nėra lengviausia užduotis, nes susiduriame su kliūtimis, aprašytomis anksčiau. Jeigu palyginsime kaip sekėsi kitiems neuroninių tinklų modeliams atpažinti būtent šį duomenų rinkinį, turime tokius rezultatus:
90.4\% , kur neuronų tinklų modelį sukūrė mokslininkai iš Kinijos Mokslų ir Technologijų universiteto. \cite{Person_reindentification}
\begin{figure}[H]
\centering
\includegraphics[scale=1.0]{img/Frame_diff.png}
\caption{Žmogaus skiringi eisenos kadrai} % Antraštė įterpiama po paveikslėlio
\label{img:mlp}
\end{figure}
Taip pat, žemiau pateikiu kitus rezultatus, gautus su kitais neuroniais tinklų modeliais:

\begin{center}
\begin{tabular}{ | m{7em} | m{5em}| } 
\hline
Modelis & Tikslumas \\
\hline
Spindle & 94.4\% \\
\hline
PSE & 86.6\% \\
\hline
Part-Aligned& 94.4\% \\
\hline
AACN & 96.7\% \\
\hline
\end{tabular}
\captionof{table}{CUHK01 duomenų tikslumo rezultatai}
\end{center}

\pagebreak

\subsection{Vaizdai su prastais rezultatų palyginimais}
Visgi ne visi vaizdai buvo tesingai palyginimi ir nustatomi trejetų tinklo modelio. Todėl šiame darbe buvo ieškoma ir nagrinėjama, kurios vaizdų įvestis turėdavo prastus rezultatus bei buvo bandoma išsiaiškinti kas sukeldavo prastus rezultatus.

\begin{figure}[H]
\centering
\includegraphics[scale=1.0]{img/Wrongly_detected.png}
\caption{Blogai identifikuotos vaizdų poros} % Antraštė įterpiama po paveikslėlio
\label{img:mlp}
\end{figure}

Dauguma nuotraukų, kurios buvo blogai identifikuotos skirėsi viena nuo kitos spalvų intensivumu, skirtingomis spalvomis. Pavyzdžiui matome, kad tiek pirmoje ir antroje nuotraukų porose įsiterpusi nauja žalia spalva (žalumos objektai) leidžia manyti, kad tai yra viena iš priežasčių, kodėl nepavyko tinkamai identifikuoti to paties asmens kadrus. Taip pat antroje poroje matome, kad žmogaus judėjimo kampas į objektyvą skiriasi, o ir nuotraukų šviesos intensyvumas yra kitoks. Todėl tai leidžia daryti prielaida, kad ir vaizdų palyginimo rezultatai bus prasti, jei nuotraukos bus nufotografuotos ne kokybiškai - apšvietimo skirtumai, netvarkingas fonas ar atsiradusi okliuzija \cite{Person_reindentification}.
\pagebreak

\begin{figure}[H]
\centering
\includegraphics[scale=1.0]{img/image_diff_examples.png}
\caption{Sunkiai identifikuojamų nuotraukų poros} % Antraštė įterpiama po paveikslėlio
\label{img:mlp}
\end{figure}
Viršuje esančios žmonių nuotraukų poros yra sudėtingai identifikuojamos dėl įvairių priežasčių: a) dėl skirtngo objektyvo kampo, b) skirtingos žmogaus pozos, c) prastos nuotraukos kadro užfiksavimo (susiliejas vaizdas), d) ne vienodame lygyje esančios žmogaus kūno dalis, atsiradusios dėl jo judėjimo, e) netvarkingas fonas, f) okliuzija.

\subsection{Sprendimo būdai}
Visgi ieškant literatūroje informacijos, kurioje būtų aprašyti sprendimo būdai, kai lyginami vaizdai vienas su kitu yra nekorektiški bruožų atžvilgiu, tačiau panašūs ar priklauso tai pačiai klasei, yra siūlomi keli sprendimų būdai. Pavyzdžiui, ypatybių fokusavimą nukreipiant į lokalias detales buvo sukurtas nesudėtingas suskirstymas asmens nuotraukos į kelias fiksuotas ir nekintančias dalis, pavyzdžiui į horizontales juosteles ir taip mokantis lokalių nuotraukų bruožų atpažinimo. Visgi, kaip tyrimai iš literatūros nurodė toks skirstymas prastai lygina identifikuojamo asmens kūno dalis \cite{Person_reindentification}. Kituose tyrimuose ir bandymuose buvo bandoma naudoti žmogaus pozą, tokiu būdu identifikuojant skirtingas kūno dalis: kojas, rankas, veidą. Tačiau ir toks atpažinimo metodas yra prastas, kad gauti norimus rezultatus, nes identifiuoti tas pačias žmogaus kūno dalis iš skirtingų žmogaus padėčių kelia problemų dėl asmens skirtingos judėjimo pozos nuotraukos \cite{Person_reindentification}.
\pagebreak
\section{Rezultatai ir išvados}
\thispagestyle{empty} 
\subsection{Rezultatai}
\begin{enumerate}
\item{Buvo lyginimi kiti trejetų tinklai rinkoje, nagrinėjami jų rezultatai.}
\item{Buvo pasiūlytas trejetų tinklų modelis, kuris sugebėjo teisingai atpažinti 85.7\% vaizdų, palyginus su kitais rinkoje esančiai modeliais, kurių rezultatai sviruoja nuo 86.6\% iki 96.7\%, pasirinktas modelis turėjo konkuruojančius rezultatus, nors ir ne pačius geriausius.}
\item{Buvo išskirtos metrikos pagal ką reitinguoti vaizdų atpažinimą.}
\item{Aprašyti tyrimo rezultatai, rastos problemos, dėl kurių yra sudėtinga labai gerai atpažinti nuotraukos, viena iš priežasčių, žmogaus eisenos trajetorijos skirtumai tarp lyginamų vaizdų.}
\end{enumerate}
\subsection{Išvados}
\begin{enumerate}
\item{Yra sudėtinga lyginti trejetų ir Siamo modelius, nes jų architektūra gali varijuota pagal tai kaip modelis yra įgyvendintas. Teisingiau yra lyginti specialius jų sukurtus modelius vienas su kitu.}
\item{Pasirinktas trejetų tinklų modelis parodė neblogus rezultatus identifikuojant žmonių eisenos kadrus.}
\item{Identifikuojant žmones nuotraukose susiduriama su rimtomis problemomis, kai to paties žmogaus nuotraukos skiriasi dėl pašalinių priežasčių. Kas lemia kad žmonių identifikacija yra sudėtingas procesas, kuris tam tikroms nuotraukoms neturi sprendimo būdų.}
\end{enumerate}
\subsection{Ateities darbai}
\begin{enumerate}
\item{Išbandyti kitą trejetų modelį su pasiūlytais būdais, kurie gebėtų dar geriau atpažinti žmonių siluetus}
\item{Atlikti atpažinimo rezultatų analizę su kitis neuronų tinklais.}
\item{Ištestuoti su daugiau duomenų rinkinių. Stebėti ir lyginti rezultatus}
\item{Iškelti savo palyginimo metodą, kuris gebėtų teisingai identifikuoti žmonių siluetus nuotraukose.}
\end{enumerate}
\pagebreak

\section{Priedai}
\thispagestyle{empty} 
\subsection{Žodynas}
\begin{itemize}
\item{Svoris(angl. weight) - parametras neuroniname tinkle, kuris transformuoja įvesties duomenys paslėptuose sluoksniuose.}
\item{Nuostolių funkcija(angl. loss function) - funkcija, padedanti optimizuoti svorius, taip sumažinant neatitikimų nuostolius.}
\item{Detalieji vaizdai(angl. fine-grained) - detalūs vaizdai. Vaizdo klasifikavimo užduotyse, tai yra įvesties vaizdai, kuriuos yra sudėtinga išskirti klasėms, pvz.: identifikuojant skirtingų markių automobilius.}
\item{Aktyvacijos funkcija(angl. activation function) - funkcija, skirta nustatyti ar neuronas turi būti aktyvuotas, skaičiuojant svorių sumą bei pridedant postūmio parametrą.}
\item{Postūmis(angl. bias) - papildomas parametras neuroniniuose tinkluose, kuris padeda koreguoti išvesti kartu su svorių įvesties suma, skirta neuronams perduoti. Taip pat šis parametras leidžia perstūmti aktyvacijos sumą nuo kairės į dešinę.}
\end{itemize} 
\pagebreak

\printbibliography[heading=bibintoc] 
\thispagestyle{empty}

\end{document}

